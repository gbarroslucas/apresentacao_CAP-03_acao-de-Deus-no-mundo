\section{Introdução}

\begin{frame}{Esclarecimentos Iniciais}
 \centering
 \includegraphics<2->[width = 0.5\textwidth]{chess}
 
 \begin{minipage}{0.8\textwidth}
  \begin{exampleblock}<3->{\textbf{Possível Conflito}}
   \justifying
    As ações de Deus no mundo (os milagres) são incompatíveis com a ciência
    contemporânea.
  \end{exampleblock}
 \end{minipage} 
\end{frame}

\begin{frame}<1-11>[label=mundos]{Esclarecimentos Iniciais}{O que pensa o Cristão?} %-----> marcação para esclarecimentos
 \begin{itemize}
  \item<2-|alert@2> Deus é uma Pessoa;
   \only<3-5>{
    \begin{itemize}
     \item<3-5> ser dotado de conhecimento e afeição;
     \item<4-5> ser com objetivos definidos;
     \item<5> age para alcançar tais objetivos.
    \end{itemize}
   }
  \item<6-|alert@6> Deus é Onisciente, Onipotente, Onipresente e Completamente Bom;
  \item<7-|alert@7> Deus é um ser necessário;
   \only<8-13>{
    \begin{itemize}
     \item<8-> existe em todos os ``mundos possíveis''
      \begin{itemize}
       \item<9-> alternativas que Deus tinha para criar o mundo (Leibniz)
       \item<10-> formas como as coisas poderiam ter sido, se elas fossem diferentes.
       \item<11-> descrição consistente da realidade
      \end{itemize}
     \item<12-> Verdades Necessárias
      \begin{itemize}
       \item<13> verdades em todos os mundos possíveis.
      \end{itemize}
    \end{itemize}
   }
  \item<14-|alert@14> Deus criou o mundo;
  \item<15-|alert@15> Deus conserva o mundo;
   \only<16-18>{
    \begin{itemize}
     \item<16-> alguns veem como ``recriação'';
     \item<17-> \textcolor{NordYellow}{convir} de Deus e o mundo
      \begin{itemize}
       \item<18-> Deus ``permite'' toda ocorrência causal
      \end{itemize}
    \end{itemize}
   }
  \item<19-|alert@19> Nada é ``mero acaso''.
   \begin{itemize}
    \item<20-> há regularidade e previsibilidade
    \item<21-> Deus, às vezes, age de forma diferente (trata diferente sua criação)
     \begin{itemize}
      \item<22-> \textcolor{NordYellow}{ação particular} (vai além da criação e conservação.)
     \end{itemize}
   \end{itemize}
 \end{itemize}

\end{frame}

\begin{frame}{\textbf{Mundos possível?}}
  \begin{exampleblock}{Em termos proposicionais}
   \begin{itemize}
    \item<2-> Nosso mundo (mundo real):
     \only<3->{
      \[
       p_1 \wedge p_2 \wedge p_3 \wedge \ldots \wedge p_n ;
      \]
     }
    \item<4-> Outros mundos: qualquer combinação que negue o mundo real.
      \begin{align*}
       \only<5->{\textcolor{NordYellow}{(\sim\! p_1)} \wedge p_2 \wedge p_3 \wedge &\ldots \wedge p_{n-1}\wedge p_n\\}
       \only<6->{p_1 \wedge \textcolor{NordYellow}{(\sim\! p_2)} \wedge p_3 \wedge &\ldots \wedge p_{n-1}\wedge p_n\\}
       \only<7->{p_1 \wedge p_2 \wedge \textcolor{NordYellow}{(\sim\! p_3)} \wedge &\ldots \wedge p_{n-1}\wedge p_n\\}
       \only<8->{&\;\;\vdots}
      \end{align*}
   \end{itemize}
  \end{exampleblock}
\end{frame}

\begin{frame}{Mundos possíveis?}
	 \begin{exampleblock}{Exemplos possíveis}
   \begin{itemize}[<+->]
    \item Mundo real:     \textcolor{NordYellow}{Asaph não caiu da \textit{bike} descendo a ladeira.}
    \item Mundo Possível: \textcolor{NordYellow}{Asaph caiu da \textit{bike} a ladeira.}
    \item Mundo real:     \textcolor{NordGreen}{Glênon é um professor carrasco de física.}
    \item Mundo possível: \textcolor{NordGreen}{Glênon não é um professor carrasco de física.}
    \item Mundo real:     \textcolor{NordYellow}{Cachorros não falam ou escrevem}
    \item Mundo possível: \textcolor{NordYellow}{Cachorros falam e escrevem}
    \item Mundo real:     \textcolor{NordGreen}{Agora está chovendo.}
    \item Mundo Possível: \textcolor{NordGreen}{Agora não está chovendo.}
   \end{itemize}
  \end{exampleblock}
  
  \begin{block}<9->{Pergunta importante}
   \only<10->{$\triangleright$ É possível, em qualquer mundo, está chovendo e não chovendo?}
  \end{block}
\end{frame}

\begin{frame}{Falsidades Necessárias}
	 \begin{exampleblock}<2->{Exemplos de falsidades em qualquer mundo possível}
   \begin{itemize}
    \item<3-> $ p \wedge (\sim\! p)$;
    \item<4-> A ideia de que ``$ 1 + 1 \neq 2 $''.
    \item<5-> Um objeto que possui qualidades maximais em certo conjunto, mas que não contenha alguma qualidade desse conjunto.
   \end{itemize}
  \end{exampleblock}
  
  \begin{itemize}
   \item<6-> Qual o contrário de Falsidades Necessárias?
    \begin{itemize}
     \item<7-> Tautologias $\big(\text{e.g.: } p \vee (\sim\! p)\big)$: Está chovendo ou não chovendo.
     \item<8-> A ideia de que ``$ 1 + 1 = 2 $''
    \end{itemize}
   \item<9-|alert@9> É possível um ser maximal existir em um mundo possível e não existir em outro?
   \item<10-> Falsidades Necessárias \textit{vs} \textcolor{NordGreen}{Verdades Necessárias}
  \end{itemize}
\end{frame}

\againframe<12->{mundos} %----------------> volta à marcação do frame principal