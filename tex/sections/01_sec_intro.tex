\section{Introdução}

\begin{frame}{Esclarecimentos Iniciais}
 \centering
 \includegraphics<2->[width = 0.5\textwidth]{chess}
 
 \begin{minipage}{0.8\textwidth}
  \begin{exampleblock}<3->{\textbf{Possível Conflito}}
   \justifying
    As ações de Deus no mundo (os milagres) são incompatíveis com a ciência
    contemporânea.
  \end{exampleblock}
 \end{minipage} 
\end{frame}

\begin{frame}{Esclarecimentos Iniciais}{O que pensa o Cristão?}
 \begin{itemize}
  \item<2-|alert@2> Deus é uma Pessoa;
   \only<3-5>{
    \begin{itemize}
     \item<3-5> ser dotado de conhecimento e afeição;
     \item<4-5> ser com objetivos definidos;
     \item<5> age para alcançar tais objetivos.
   \end{itemize}
   }
  \item<6-|alert@6> Deus é Onisciente, Onipotente, Onipresente e Completamente Bom;
  \item<7-|alert@7> Deus é um ser necessário;\
   \only<8-13>{
    \begin{itemize}
     \item<8-> existe em todos os ``mundos possíveis''
      \begin{itemize}
       \item<9-> alternativas que Deus tinha para criar o mundo (Leibniz)
       \item<10-> formas como as coisas poderiam ter sido, se elas fossem diferentes.
       \item<11-> descrição consistente da realidade
      \end{itemize}
     \item<12-> Verdades Necessárias
      \begin{itemize}
       \item<13> verdades em todos os mundos possíveis.
      \end{itemize}
    \end{itemize}
   }
  \item<14-|alert@14> Deus criou o mundo;
  \item<15-|alert@15> Deus conserva o mundo;
   \only<16-18>{
    \begin{itemize}
     \item<16-> alguns veem como ``recriação'';
     \item<17-> \textcolor{NordYellow}{convir} de Deus e o mundo
      \begin{itemize}
       \item<18-> Deus ``permite'' toda ocorrência causal
      \end{itemize}
    \end{itemize}
   }
  \item<19-|alert@19> Nada é ``mero acaso''.
   \begin{itemize}
    \item<20-> há regularidade e previsibilidade
    \item<21-> Deus, às vezes, age de forma diferente (trata diferente sua criação)
     \begin{itemize}
      \item<22-> \textcolor{NordYellow}{ação particular} (vai além da criação e conservação.)
     \end{itemize}
   \end{itemize}
 \end{itemize}

\end{frame}
